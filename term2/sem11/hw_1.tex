\documentclass{article}
\usepackage{amsmath, amssymb}
\usepackage[russian]{babel}

\begin{document}

\section*{Решение}

\noindent \textbf{Дано:}
\begin{itemize}
    \item Ориентированный граф $G = (V, E)$ с истоком $s \in V$ и стоком $t \in V$.
    \item Задан максимальный поток $f$ из $s$ в $t$.
    \item Существует единственный минимальный разрез $(S, T)$, где $s \in S$, $t \in T$.
\end{itemize}

\noindent \textbf{Вопрос:} Все ли вершины достижимы из $s$ в остаточной сети $G_f$?

\subsection*{1. Свойства остаточной сети и минимального разреза}
Пусть $R$ --- множество вершин, достижимых из $s$ в остаточной сети $G_f$. 
По теореме о максимальном потоке и минимальном разрезе:
\begin{itemize}
    \item $(R, V \setminus R)$ является \emph{некоторым} минимальным разрезом.
    \item Все рёбра $(u, v)$ в исходном графе, где $u \in R$, $v \in V \setminus R$, насыщены потоком $f$, и в $G_f$ прямых рёбер из $R$ в $V \setminus R$ нет.
\end{itemize}

\subsection*{2. Единственность минимального разреза}
Поскольку минимальный разрез $(S, T)$ \textbf{единственный}, для любого максимального потока $f$ множество $R$ должно совпадать с $S$. 
В противном случае существовал бы другой минимальный разрез $(R, V \setminus R)$ с той же пропускной способностью, что противоречит единственности.

Следовательно, в остаточной сети $G_f$:
\[
R = S.
\]
То есть из $s$ достижимы в точности вершины множества $S$, и никакие вершины из $T = V \setminus S$ не достижимы из $s$.

\subsection*{3. Достижимость вершин из $T$}
Множество $T$ непусто, так как содержит сток $t$. Поэтому:
\begin{itemize}
    \item Вершина $t$ не достижима из $s$ в $G_f$.
    \item Все остальные вершины $T$ также не достижимы из $s$.
\end{itemize}

\subsection*{4. Ответ}
Утверждение <<все вершины достижимы из $s$ в остаточной сети $G_f$>> \textbf{неверно}. 
Контрпримером является хотя бы сток $t$, который лежит в $T$ и недостижим.

\[
\boxed{\text{Нет, не все вершины достижимы из } s \text{ в остаточной сети } G_f.}
\]

\end{document}